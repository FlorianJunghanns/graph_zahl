%!TEX root = ../main.tex

\begin{frame}
\frametitle{Storage}
\begin{itemize}
%\item \emph{How is data stored physically, i.e., file format and organization, data catalog, etc.?}
%\item \emph{Where can data be stored physically, i.e., disk or file storage, in-memory (RAM), flash or SSD, traditional database, cloud storage (GFS, HDFS, S3), etc.?}
%\item \emph{How is data accessed, i.e., how do read and write operations work?}
\item Database can be spread over several servers with the help of "`Storage Locations"' and several other configurations
\item In-memory capable as well as persistent on hard drive
\item If on hard drive: files in a configurable directory
\item Edges and vertices are stored in seperate files
\end{itemize}
\end{frame}

\begin{frame}
\frametitle{Storage}
\begin{itemize}
\item Files consist of containers
\item Containers consist of storage pages
\item Containers are the smallest entity to lock
\item Storage pages are the smallest entity for read/write
\end{itemize}
\end{frame} 
